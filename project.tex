\documentclass{iansnotes}

\title{Project: Theory of Algorithms}
\author{ian.mcloughlin@atu.ie}
\date{Summer 22/23}

\begin{document}
 
\maketitle

These are the instructions for the project for Theory of Algorithms in Summer 22/23.
The project covers 30\% of the overall assessment of this module.


\section{Submission}

\begin{itemize}
  \item The deadline for submission is March 31\textsuperscript{st}, 2023. 
  \item Your whole submission must be in a single GitHub repository.
  \item Use the form on the Moodle page to submit your repository.
  \item All you need to do is submit the repository URL.
  \item You should submit the URL as soon as possible.
  \item Commits in GitHub on or before the deadline will be considered.\footnote{Once you have submitted your URL, you do not need to do anything other than commit to your repository and push the changes to GitHub.}
\end{itemize}


\section{What to submit}

\begin{itemize}
  \item You must create a GitHub repository solely for this assessment.
  \item Your repository must contain a single Jupyter notebook called \texttt{polynomial\_time.ipynb} on the topic of The Polynomial Time Complexity Class ($\textsf{P}$).
  \item You should imagine your classmates are the target audience for your work, explaining all concepts in terms you imagine they would understand.
  \item The notebook should contain text, code, plots, mathematical notation, graphics, and any other items useful in explaining the concepts.
  \item The notebook should be clean, clear, and visually appealing.
  \item Your GitHub repository must contain \texttt{README.md} and \texttt{.gitignore} files with appropriate contents typical of a GitHub repository.
  \item Your repository should not contain any unnecessary content such as temporary files.
  \item Your repository commit history should show regular and reasonably sized commits.
\end{itemize}


\section{Marking Scheme}
Your submission will be marked using the four categories below.
To receive a good mark in a category, your submission needs to provide evidence of meeting each of the criteria listed under it\footnote{In line with ATU policy, the examiners' overall impression of the submission may affect individual marks in each category.}.

\begin{description}
  \item[Research $(25\%)$:] evidence of research on topics; appropriate referencing; building on work of others; comparison to similar work.
  \item[Development $(25\%)$:] clear, concise, and correct code; appropriate tests; demonstrable knowledge of different approaches and algorithms; clean architecture.
  \item[Documentation $(25\%)$:] clear explanations of concepts in notebooks; concise comments in code and elsewhere; appropriate, standard README for a GitHub repository.
  \item[Consistency $(25\%)$:] tens of commits, each representing a reasonable amount of work; literature, documentation, and code evidencing work on the assessment; evidence of reviewing and refactoring.
\end{description}

 
\section{Advice}

\begin{itemize}
  \item Students sometimes struggle with the freedom given in an open-style assessment.
  \item You must decide where and how to start, what is relevant content for your submission, how much is enough, and how to make the submission your own.
  \item This is by design --- we assume you have a reasonable knowledge of programming and an ability to source your own information.
  \item Companies tell us they want graduates who can (within reason) take initiative, work independently, source information, and make design decisions without needing to ask for help.
  \item The point of this assessment is to demonstrate you can do that.
  \item You need a plan, you cannot just start coding straight away.
\end{itemize}


\section{Policies}

\begin{itemize}
  \item You are bound by all ATU policies and any GMIT policies that have not yet been replaced by new ATU policies.
  \item Review the GMIT Quality Assurance Framework~\autocite{gmitqaf}.
  \item Pay particular attention to the Policy on Plagiarism and the Code of Student Conduct.
  \item If you have any doubts about what is permissible, email me to ask\footnote{\url{ian.mcloughlin@atu.ie}}.
\end{itemize}

\end{document}